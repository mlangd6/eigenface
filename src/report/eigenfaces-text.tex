\documentclass[12pt,french]{article}

\usepackage{amssymb}
\usepackage{babel}

\title{Eigenfaces}
\author{
Bouarah Romain \and
Langdorph Matthieu \and
Ketels Lucas \and
Souffan Nathan
}


\begin{document}
\maketitle
\newpage

\part{Partie Mathématiques}


\section{Calcul des Eigenfaces}

\subsection{Travail dans $\mathbb{R}^{N \times N}$}
Considérons une image de visage comme une matrice $N \times N$ dont le coefficient $(i,j)$ est égal au niveau de gris du pixel $(i,j)$ (l'origine se situant dans le coin haut gauche).  
On transforme ensuite cette matrice comme un vecteur de $\mathbb{R}^{N \times N}$ en juxtaposant les colonnes l'une en dessous de l'autre, par exemple.

\subsection{Calcul des valeurs propres et des vecteurs propres de la matrice de covariance}
Les images des visages sont globalement similaires, donc ces images ne seront pas distribuées aléatoirement dans notre espace $\mathbb{R}^{N \times N}$.
On peut donc décrire notre espace des visages de manière plus fine (\textit{i.e.} avec moins de dimensions).

\subsubsection{Matrice de covariance}
Avant de commencer la réduction de l'espace de travail, nous allons d'abord calculer la matrice de covariance.\\
Supposons que nous avons $M$ images de visage qu'on note $\Gamma_1,~\Gamma_2,~\dots,~\Gamma_M$. On a $\Psi = \frac{1}{M}\displaystyle\sum_{i=1}^{M} \Gamma_i$ correspondant à la moyenne des visages.
Chaque visage différe donc de la moyenne par le vecteur $\Phi_i = \Gamma_i - \Psi$.
\\ \emph{Définir la matrice de covariance}\\
La matrice de covariance est $C = \frac{1}{M}\displaystyle\sum_{i=1}^{M} \Phi_i \Phi_i^T=AA^T$ où la matrice $A = [\Phi_1 \Phi_2 \dots \Phi_M]$

\subsubsection{Méthode 1: Analyse en composantes principales}
\subsubsection{Méthode 2: Décomposition en valeurs singulières}
\section{Utilisation des eigenfaces pour classer une image de visage}
\subsection{Projection dans l'espace des visages}
\subsection{Analyse de la projection}









\end{document}
